\chapter{Comparing to human adaptation} \label{chapter:human}

In the previous chapter we proposed a framework for modeling human adaptation to new tasks, based on relationships between tasks. To understand whether we have developed a good model, it is necessary to compare its adaptation to human adaptation. It is worth stopping for a moment to consider how flexible humans actually are. We certainly experience short-term interference from switching tasks or goals \citep{Rogers1995}. Over longer periods of time, our adaptation might be a response to learning in the new situation, rather than the type of zero-shot flexbiility we modeled. How flexibly are we able to adapt to task changes? \par
Unfortunately a complete evaluation of human flexibility is a large research program. Thus, in this chapter we evaluate human adaptation in a task inspired from those used as demonstration in the previous chapter. In particular, we implemented a more human-comprehensible version of the poker-like card game we used for those experiments, and its losing variation. \par

\section{Experimental details}
We made several changes to the game to make it easier for humans to learn. We also 

See appendix \ref{appendix:human} for detailed methods.


\section{Human performance}

\section{Comparing HoMM to humans}
