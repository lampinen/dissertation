\chapter{Conclusions \& looking ahead} \label{chapter:conclusions}

While artificial intelligence has achieved incredible things in recent years, there are still immense differences between artifical and natural intelligence. In this dissertation, I have focused on how humans are able to reuse our knowledge flexibly in new tasks, before we have any data on them. I have suggested that one mechanism underlying this is an ability to transform prior task representations to adapt them to the new task. I have proposed meta-mapping -- higher-order tasks that transform task representations -- as a plausible computational mechanism for this adaptation. I have provided a parsimonious implementation of this framework in the form of Homoiconic Meta-Mapping. I have demonstrated the effectiveness of HoMM by showing its zero-shot task performance across a wide variety of domains, from polynomial regression to visual classification and reinforcement learning. HoMM is often able to achieve 90\% performance on a new task with no data on that task at all. This brings deep learning models a step closer to human-like flexibility. It therefore has implications for both cognitive science and artificial intelligence. \par  

\section{On flexibility in minds and machines}

In the introduction, I noted how researchers in cognitive science have aggressively critiqued deep learning for its lack of flexibility \citep[e.g.][]{Lake2015, Lake2016, Lake2017, Marcus2018}. We have addressed one challenging aspect of flexibility in this work -- the ability to take our knowledge of a task, and adapt to some variation, such as achieving an orthogonal goal. While this might be challenging for standard deep-learning models, our general framework of meta-mapping makes it possible. Thus, at their most basic level, my results present a challenge for those who would say that deep learning models are inflexible, and therefore are inappropriate models for cognitive processes. \par  
Indeed, we see our work as following more in the tradition of work that explores how systematic, structured generalization can emerge from the structure of learning experience, without needing to be built into the model \citep{McClelland2010a, McClelland2010, Hansen2017}. Without building in compositional representations of tasks, our model can learn to exploit the shared structure in the concept of ``losing'' across a few card games to achieve 85\% performance in losing a game it has never tried to lose before. There are a number of potential benefits to letting the compositional structure emerge --- for example, it could potentially allow for novel decompositions at test time. The ability of our model to switch to losing a game it has never tried to lose before is somewhat like inferring a novel decomposition, but we would hope that a model with richer training data could make much larger inferential leaps. The ability of our model to perform well on held-out meta-mappings supports this hope, but further work will be needed to verify it. \par
However, the type of flexibility we have proposed is still limited. It requires exactly identifying the task to be adapted, and when that adaptation should occur (i.e. where the task boundary is). It would be more realistic to relax these assumptions, by combining with techniques that have been used for similar purposes in meta-learning \citep[e.g.][]{Nagabandi2019}. Futhermore, while we showed in some cases that the meta-mapping could be specified by language, we often specified it by examples of the task mapping. This does not necessarily mean that using language in the other settings would not work -- it is simply that the time and computational resoucess I had available to me while working on this project were finite. It is likely that language could mediate meta-mapping in and the cueing of examples in more settings and more complex ways than we have explore here.\par 
Indeed, an interesting future direction would be to consider how meta-mapping and language can mutually constrain one another when adapting to a new situation, and how they can interact with learning from examples in more complex ways than we explored in Ch. \ref{chapter:timescales}. Cognition is complex, and any single model is guaranteed to be an oversimplified approximation of human cognitive processes in real-world situations. Our results should not be taken as a suggestion that meta-mapping is the only cognitive mechanism for adaptation. Instead, our results demonstrate that meta-mapping may be useful as one tool for building models with more human-like adaptibility. \par
With these caveats in mind, however, I believe that meta-mapping offers a way to create models that can bring the intelligence of deep learning systems closer to the flexibility of the human mind. While we have demonstrated these results in some simple settings, one of the powerful features of deep learning is that its results tend to improve as datasets grow more complex and realistic \citep{Hill2019a}. We hope that our research will help guide the way to building even more flexible models in more realistic domains. 

\section{Relating to cognitive science}

Our model provides a tool for modeling human adaptibility, which has many potential direct applications. However, it also opens up possibilites for exploring abstraction and recursion in cognition more broadly. It would be interesting to explore concepts can be recursively built upon other concepts, as happens in learning of mathematical concepts \citep{Wilensky1991, Hazzan1999, Lampinen2017b}. For example, addition can be seen as repeated succession, multiplication can be seen as repeated addition, exponentiation as repeated multiplication, and this process is recursively continued in up-arrow notation. A homoiconic system like HoMM seems closer to being able to capture this recursive definition. \par 
Furthermore, some of the inspiration for my work comes from long-standing ideas in cognitive science about how humans re-represent their knowledge into more generalizable forms \citep{Karmiloff-Smith1986,Clark1993}. It would be interesting to explore whether unsupervised learning over task representations and meta-mappings could model some of these phenomena.\par 

\section{Relating to artificial intelligence}

There are a number of potential direct applications in artificial intelligence, including applying meta-mapping in more realistic settings such as robotics. Domains like robotics are especially interesting from the meta-mapping perspective, because exploration in real world settings is costly and must be safe \citep{Turchetta2016}, and so the substantial reduction in errors made when using meta-mapping as the starting point for learning a new task may be valuable. \par 

Applying meta-mapping to different types of adaptation in RL also opens many possibilites, especially in combination with model-based methods. 

More generally, there are a number of aspects of the approach that could be altered. First, although we used HyperNetworks to parameterize our task network, it would also be reasonable to have a fixed task network which simply receives the task representation as an additional input. We tried this in some of our domains, and found it did not perform as well, but it might be a useful approach in some settings. We also noted in the visual cagegories domain that linear task networks seemed to improve meta-mapping, while nonlinear ones seemed to result in better basic task performance --- thus it might be reasonable to consider a deep, nonlinear task network, but with a linear skip-connection from beginning to end. Furthermore, cognitive processing is much more complex than our model, and replacing the feed-forward task network with a recurrent network, or even a more complex architecture, such as the Differentiable Neural Computer \citep{Graves2016} would likely increase the ability of the model to perform and adapt on complex tasks. These are simply examples of the many architectural choices that could be further explored. These are simply examples of the many architectural choices that could be further explored. \par


\section{Future directions}





\section{Looking ahead}


