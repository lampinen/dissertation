\lstset{language=Python,
    frame=single,
    breaklines=true,
    postbreak=\raisebox{0ex}[0ex][0ex]{\ensuremath{\color{red}\hookrightarrow\space}}
}

\restylefloat{figure}
\newcommand{\Prop}{\textbf{Proposition: }}
\newcommand{\Prob}{\textbf{Problem: }}
\newcommand{\Prf}{\textbf{Proof: }}
\newcommand{\Sol}{\textbf{Solution: }}
\newcommand{\grad}{\nabla}
\newcommand{\Nats}{\mathbb{N}}
\newcommand{\Ints}{\mathbb{Z}}
\newcommand{\Rats}{\mathbb{Q}}
\newcommand{\Reals}{\mathbb{R}}
\newcommand{\Comps}{\mathbb{C}}
\newcommand{\Prb}[1]{P\left( #1 \right)}
\newcommand{\PT}[1]{P\left( \text{#1} \right)}
\newcommand{\PCon}[2]{P\left( #1 \mid #2 \right)}
\newcommand{\PConT}[2]{P\left( \text{#1} \mid \text{#2} \right)}
\DeclareMathOperator{\E}{\mathbb{E}}
\DeclareMathOperator{\tr}{\textbf{tr}}
\DeclareMathOperator*{\argmin}{argmin}
\DeclareMathOperator*{\argmax}{argmax}
\newcommand{\thus}{\quad\mathlarger{\mathlarger{\mathlarger{\Rightarrow}}}\quad}
\newcommand{\wght}{\mathbf{w}}
\newcommand{\im}{\text{im }}
%%\newcommand\scalemath[2]{\scalebox{#1}{\mbox{\ensuremath{\displaystyle #2}}}}


%% Tikz stuff
\usetikzlibrary{shapes,arrows}
\usetikzlibrary{decorations.pathreplacing, decorations.pathmorphing, snakes, calligraphy}

\tikzstyle{block} = [rectangle, draw, thick, align=center, rounded corners]
\tikzstyle{boundingbox} = [very thick, gray]
\tikzstyle{dashblock} = [rectangle, draw, thick, align=center, dashed]
\tikzstyle{conc} = [ellipse, draw, thick, dashed, align=center]
\tikzstyle{netnode} = [circle, draw, very thick, inner sep=0pt, minimum size=0.5cm]
\tikzstyle{relunode} = [rectangle, draw, very thick, inner sep=0pt, minimum size=0.5cm]
\tikzstyle{line} = [draw, very thick, -latex']
\tikzstyle{arrow} = [draw, ->, thick]
\tikzstyle{mapsto} = [draw, |->, thick]

\def\checkmark{\tikz\fill[scale=0.4](0,.35) -- (.25,0) -- (1,.7) -- (.25,.15) -- cycle;}


%% Some colors stolen from ColorBrewer
\definecolor{bpurp}{HTML}{984ea3}
\definecolor{bblue}{HTML}{377eb8}
\definecolor{bgreen}{HTML}{4daf4a}
\definecolor{borange}{HTML}{ff7f00}
\definecolor{bred}{HTML}{a50f15}

% footnote without a marker
\makeatletter
\def\blfootnote{\gdef\@thefnmark{}\@footnotetext}
\makeatother

