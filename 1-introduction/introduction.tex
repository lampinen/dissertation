\chapter{Introduction} \label{chapter:introduction}
\epigraph{``The great evolutionary advantage of the human species is adaptability.''}{Dedre Gentner, \textit{Why We're So Smart}}
\epigraph{``The most elementary single difference between the human mind and that of brutes lies in this deficiency on the brute's part to associate ideas by similarity.''}{William James, \textit{Principles of Psychology}}
Deep learning models have achieved incredible success recently, reaching human-level (or super-human) performance in domains ranging from vision \citep{Szegedy2015} to board games \citep{Silver2016} and video games \citep{Vinyals2019}. However, these models still lack some important human abilities \citep[e.g.][]{Lake2016}. These models are often data-hungry, while humans can frequently learn from relatively few examples. Furthermore, even if deep-learning models are given a large quantity of training data, they may not be able to generalize well outside the data distribution they experience during training. Finally, once knowledge is learned in the weights of a deep learning model, it is difficult to flexibly reuse that knowledge. From the perspectives of Gentner \& James (quoted above), most deep learning models are unlike humans. Deep learning models don't understand the relationship between new situations and their prior experiences, and so do not know how to flexibly adapt to changes in their environment or goals. \par
By contrast, humans can reuse knowledge flexibly. We can learn from few examples. We can introspect about our learned behavior to explain or change it. We are often able to accurately behave according to linguistic instructions, without requiring examples of the desired behavior. We can even remap our behavior in a way completely inconsistent with our prior behavior, such as trying to lose a game we have previously been trying to win \citep{Lake2016}. All of these types of flexibility can be quite difficult for deep learning systems. \par
This apparent contrast between humans and neural networks has caused researchers to question the validity of neural networks as a cognitive model for many years \citep[e.g.][]{Fodor1988}. The recent successes of deep learning have only increased the frequency of these critiques \citep[e.g.][]{Lake2015, Lake2016, Lake2017, Marcus2018}. It is important to address these perspectives, and to understand how deep learning systems can be improved to serve as better models of human flexibility. There have been a number of attempts to defend from, or integrate, these critiques, from a variety of perspectives \citep[e.g.][]{McClelland1999, McClelland2010, Hill2019a}. For example, in past work my colleagues and I have argued that these critiques overlook the benefits of transfer \citep{Lampinen2017a} and recent progress in meta-learning \citep{Hansen2017}. Recent deep learning research has made important progress on improving learning speed and generalization. I will review some of that progress below. \par
Despite this progress, however, some challenges remain. In this dissertation I will focus on the ability of humans to adapt to a new task \emph{zero-shot}, that is, without any data at all. For example, imagine trying to lose a game that you have previously been trying to win. Humans can perform reasonably on their first try at losing, despite having no data on the losing version of the task, and despite their previous goals being directly in opposition to their new one. I will suggest that we can do so by using our knowledge of what it means to lose more generally transform our representation of the present game. I will therefore provide a general computational framework for adaptating by transforming task representations, and will evaluate it across a wide range of task domains. This framework endows deep-learning models with flexibility more like that of humans. \par 
To situate my work within the broader field, in this chapter I will review some of the cognitive issues and the progress to date. I will try to provide a unifying perspective on how various types of transfer contribute to human learning and flexibility, by reviewing my prior work as well as that of others. In the remainder of this dissertation, I will tackle the problem of building deep learning models that can adapt to task alterations zero-shot. \par 

\section{Cognitive flexibility \& generalization}

What kind of flexibility do humans have? We are often able to learn rapidly. For example, we can achieve some competence in a novel video game within a few minutes \citep{Lake2016}. We can learn new concepts from seeing only a relatively small number of examples \citep[e.g.][]{Bourne1970}. We can often learn even faster if we can actively participate in the learning process by selecting examples rather than passively receiving them, especially if the concepts are simple enough that we can generate a good hypothesis space \citep{Markant2014a}. \par
We can then apply what we have learned to new situations. The studies of \citet{Bourne1970} show that, not only can people generalize to new examples of a concept they have learned, they can also transfer that knowedge to learn structurally-similar new concepts more rapidly. Indeed, even without explicit awareness of the relationship between two tasks, humans can sometimes benefit from transfer effects \citep[e.g.][]{Day2011}. In general, human analogical transfer abilities have been suggested to be a critical component of `` what makes us smart'' \citep{Gentner2003}. \par
Yet human flexibility is apparent beyond transfer between isomorphic tasks. We can often competently change our learned behavior in response to instruction or other goals, such as trying to lose a game we were previously trying to win, or trying to achieve some orthogonal task \citep{Lake2016}. Indeed, it has been known for almost a century that even other animals exhibit flexible knowledge use --- they engage in ``latent learning'' of environmental features that are irrelevant at present, but may be useful when solving future tasks \citep{Blodgett1929}. Both humans and other animals are capable of flexibly applying our knowledge in many situations. \par
However, our flexibility is not universal. Sometimes it is quite difficult for us to integrate new knowledge. For example, even undergraduate students with substantial mathematical background often struggle with understanding new mathematical concepts.\footnote{I will focus on mathematical cognition in some places within this dissertation. Although the phenomena I discuss are much more general, and I will include a number of other examples, mathematical cognition offers a unique microcosm of human abilities for transfer, abstraction and flexibility.} They may mistakenly assume the converse of a theorem, or get caught up in concrete ways of thinking about abstract concepts \citep{Hazzan1999}. Socrates' dialog about doubling the area of a square captures the misunderstandings that modern subjects make, just as it did for those over 2000 years ago, yet it does not help them to deeply understand the principle \citep{Goldin2011}. Even after engaging with the dialog, nearly 50\% of modern subjects failed at the simplest generalization of the principle: to a square of different size. Similarly, even students who complete a course in geometry in high school may not achieve formal deductive understanding of the concepts taught unless (or until) they become undergraduate mathematics majors \citep{Burger1986}. Furthermore, superficial details of how a concept is presented can have profound impacts on how easy it is to reason about, even if the underlying concept is exactly the same \citep[e.g.][]{Kotovsky1985, Kaminski2008}; we have shown that these details can therefore change how easy it is to learn future concepts in the domain \citep{Lampinen2017b}. There is a wealth of research showing that our learning is far from universally flexible. \par 
Humans also often fail to flexibly use the knowledge we have. For example, even mathematics students who can correctly state a rule or theorem are not necessarily able to apply it to create a proof \citep{Weber2001}. Similarly, even if experimental subjects can learn a basic concept rapidly, it may be difficult for them to apply it in more abstract situations or to extract more formal understanding from it \citep[e.g.][]{Lampinen2017b}. Likewise, rapid analogical transfer is often only possible when superficial details match closely, or when subjects are explicitly told to transfer \citep[e.g.][]{Gick1980}. Because of findings like these, \citet{Detterman1993} has argued that inducing transfer requires manipulations ``with the subtlety of a baseball bat,'' and so we should conclude that ``significant transfer is probably rare and accounts for very little human behavior.'' This quote is a particularly tendentious presentation of the issues, but it captures the important broader insight that humans are not always rapid learners or flexible reasoners. \par 
How can we reconcile the demonstrations of rapid learning and flexibility with the evidence that some concepts are learned slowly and some knowledge is inflexible? How can we reconcile arguments that transfer is key to ``what makes us smart'' \citep{Gentner2003}, with arguments that it ``accounts for very little human behavior'' \citep{Detterman1993}? There are a variety of factors that affect whether transfer will occur \citep{Barnett2002, Lampinen2017a}. First, we need high-quality representations of the concepts we are learning in order to reason flexibly with them. These generalizable representations are generally created through making connections between different pieces of our knowledge \citep{Wilensky1991, Schwartz2015}, and perhaps through re-representation processes \citep[][see below]{Karmiloff-Smith1992}. Second, we often need strategic meta-knowledge about where and how to apply our knowledge in new situations, which also must be learned \citep{Weber2001}. Both these factors mean that transfer may happen more easily over longer periods of time, as I have argued in my prior work \citep{Lampinen2017a}. The quality of the representations we have, and the way those representations relate to the new tasks we are presented with, both affect our ability to learn rapidly, reason flexibly, and generalize. \par 

\subsection{Flexibility as transfer}
I argue that all these types of flexibility (and inflexibility) can be seen as transfer, defined broadly as the way that ``knowledge acquired in one situation applies (or fails to apply) in other situations.'' \citep{Singley1989}. From this definition it is clear that applying learned features and structures to accelerate learning in a new situation, as in \citet{Bourne1970}, is a type of transfer. I argue that in fact all rapid learning must rely on transfer of prior knowledge in order to constrain the hypothesis space under consideration.\footnote{This conclusion is a consequence of the ``No Free Lunch'' Theorem \citep{Wolpert1996}, which shows that a learning algorithm cannot be \emph{a priori} better at generalizing than any other, including seemingly adversarial ones like choosing the model with the \emph{worst} validation error. The advantages of a learning algorithm can only be due to the match between its inductive biases and the structure of the task(s) it must learn.}\footnote{While I sometimes use terms like ``hypothesis space'' for their useful intuitions, this does not necessarily imply that I believe that reasoning is occuring explicitly in an explicit hypothesis space --- a particular set of parameters in a neural network can be seen as a particular hypothesis about an input-output mapping, and thus the parameter space of a network can be seen as a hypothesis space.} Even other types of flexibility, such as adapting to instructions, can be seen as transferring several different types of knowledge (prior knowledge about a task, other related taks, and language) to a new situation. For example, if we are asked to try to lose at chess, we are essentially presented a new task to which we need to transfer both our prior knowledge of chess and our prior knowledge of what ``trying to lose'' means. Thus flexibility and transfer are essentially two perspectives on the same broad phenomena. I will therefore use these words to refer to distinct aspects of these phenomena. Specifically. I will use ``flexibility'' to refer to the behavioral phenomena, and ``transfer'' to refer to the computational principles underlying these phenomena. \par  
Why do humans sometimes fail to adapt flexibly? There are several reasons this can occur. First, we may not have the prior knowledge necessary to be flexible. We can't transfer what we don't know. Second, our representations may lack the quality necessary to transfer them. We often can't transfer what we don't know well \citep[c.f.][]{Karmiloff-Smith1992, Hazzan1999, Weber2001}. Third, we may not recognize that we have applicable prior knowledge to transfer \citep{Detterman1993}. Finally, our prior beliefs may be inaccurate in the present situation, and so transferring them could actually interfere with our ability to learn. This effect is generally called ``negative transfer'' \citep{Singley1989}. \par
For transfer to be beneficial overall, we must generally encounter settings where our prior knowledge is applicable, and furthermore we must have good ways of integrating that prior knowledge with new experiences. In the next section I will discuss some of the features that allow us to do so. \par

\section{What factors contribute to our flexibility?}
We need to address the computational question of how and when we can use transfer to behave flexibly. In this section I will give a brief overview of the contributing factors. \par 

\textbf{Complementary learning systems:} First, it has been proposed that we have complementary learning systems \citep{McClelland1995, Kumaran2016}. These complementary systems allow us to learn rapidly from new knowledge while avoiding catastrophic interference \citep{McCloskey1989} with the statistical knowledge we have accumulated over longer timescales. The key idea is that we have a slow (parametric) learning system which sets up good representations, while a fast (nonparametric) learning system stores new knowledge by using these representations. Throughout this dissertation, we will return to this theme of mutually-supporting fast and slow learning. I will therefore divide the rest of this section into considerations of the ``slow'' and ``fast'' systems that contribute to transfer.\par
However, ``slow'' vs. ``fast'' learning systems is not a strict dichotomy. This broad distinction is useful, but in reality learning systems fall on a continuum, and the time-course of learning in a given system is often dramatically affected by what knowledge is already present \citep[][]{McClelland2013, McClelland2020}. Thus this dichotomy between learning systems should not be interpreted as a strict division. Instead, it is a useful way of highlighting the cooperation between distinct systems that operate across distinct, yet sometimes overlapping, timescales. \par 

\subsection{Slow}
In this section I will discuss the slow learning systems that contribute to transfer and flexibility. \par
\textbf{Culture \& education:} One critical type of transfer is cultural knowledge transmission. Our cultures have accumulated knowledge over extremely long time scales, and building upon this knowledge allows us to advance much more rapidly now \citep{Tomasello1993, Bengio2012}. For example, mathematical concepts have been constructed by humans \citep{Hersh1997, MacLane1986}, and it has taken us millenia to build up to domains like calculus. Yet today many students gain fluency with these ideas before they graduate high school. Because culture has set up useful representations for these concepts, we are able to acquire them much more rapidly \citep[e.g.][]{McClelland2016}. Furthermore, culture has constructed systems of education that are structured precisely to help us learn rapidly and generalize effectively. \par 
By contrast, if our culture does not represent or highlight certain concepts, we may struggle to reason about them. For example, one Amazonian tribe that lacks words for exact numbers shows substantially impaired ability to do basic tasks involving cardinality, which are interpreted as a fundamental deficit in the ability to represent exact cardinality at all \citep{Gordon2004}. Similarly, although the number line as a spatial representation of number appears relatively universal, cross-cultural and historical studies reveal that it is constructed rather than innate \citep{Nunez2011}. Culture helps set up powerful representations and metaphors for us to learn from. Without these representations, our learning would be substantially slower. \par 
Even beyond culture, various aspects of our developmental experience may construct natural curricula (perhaps discovered by evolution). For example, the structure of visual (and visuo-motor) experience shows a natural progression as children develop, with increasing complexity and diversity at older ages \citep{Fausey2016}. These natural curricula may play a critical role in our development \citep{Smith2017}.\par 
\textbf{Transfer between tasks:} Even if culture has not explicitly highlighted (or engineered) structural relationships between tasks, we can benefit from structural similarity. For example, after learning an artificial grammar, subjects can generalize their knowledge to novel sequences from the same grammar applied to novel symbols \citep[e.g.][]{Tunney2001}. From learning about simple harmonic oscillators in the context of springs, participants can transfer to a superficially unrelated problem about controlling the population of a city \citep[e.g.][]{Day2011}. Because many tasks we perform share deep underlying structures, we can take advantage of transfer to learn faster on new tasks. \par 
\textbf{Grounding, embodiment, and representation quality:} One particular type of transfer that seems to be especially useful is grounding \citep{Barsalou2007}. In particular, conceptual representations often tend to be tied into more basic perceptual-motor systems, e.g. arithmetic in the Approximate Number System \citep{Park2013}, or mathematical (and other) reasoning in gestures \citep{Goldin-Meadow1993, Goldin-Meadow1999}. Indeed, the way we talk about many important abstract concepts, such as time, seems to fundamentally rely on metaphors linking these abstract concepts to more concrete ones, such as space \citep{Lakoff1980}. This sort of grounding can be very beneficial to understanding \citep{Nathan2008, Schwartz2015, Wakefield2018}. Because our perceptual-motor system have exceptionally good representations that are trained over long developmental time-scales, we may benefit from leveraging these representations to transfer our understanding to analogous conceptual domains. \par 
However, it is also worth considering that grounding and embodiment can hold us back. Even our understanding of symbolic expressions seems to be influenced by ``meaningless'' perceptual details like their spacing \citep{Landy2007}. It has also been argued that too concrete of examples can limit generalization \citep{Kaminski2008}, although the details of that particular demonstration have been debated \citep{DeBock2011, Lampinen2017b}. There is probably some negative transfer from grounding in some cases, but overall it is probably outweighed by the positive. \par
There are also a number of conceptual issues that come along with grounding, such as how to define abstraction \citep{Dove2016}, and whether the grounding must be in the real world, or can more generally be in any concepts that are better understood \citep{Wilensky1991}, regardless of whether those concepts are basic perceptuo-motor knowledge. The line between grounding and other types of transfer can be blurry, because, like many psychological ideas, its definitions are multifarious. \par 
The overlapping field of embodied cognition has raised the opposite issue, arguing that cognition cannot be considered at all outside of the physical, physiological, and social situations in which it is grounded \citep{Anderson2003}. Others have argued that even the computation metaphor in cognition is fundamentally flawed, because it neglects the fact that intelligence evolved in systems interacting with the world \citep{Cisek1999}, through a process of hierarchically constructed control systems \citep{Cisek2019}. The fact that many of our basic concepts seemed to be formed at the optimal level for action \citep{Rosch1976} has been cited as evidence for the importance of grounding \citep{Lakoff1999}. This perspective is challenge for purely computational theories of mind \citep{fodor2001mind}. It is important not to neglect the world in which our brains reside in favor of a disembodied, computational mind. \par

\textbf{Rerepresentation and different kinds of knowledge:} The work of Karmiloff-Smith \citep[e.g.][]{Karmiloff-Smith1986, Karmiloff-Smith1992, Clark1993} focused on the idea that we repeatedly redescribe our internal knowledge, reorganizing it in order to better understand the world. To support this theory, she examined evidence of U-shaped developmental curves, where children would actually get worse at a task before they reached ceiling, often because they at first over-generalized a rule. She argues that these results show a pattern of systematic progression of knowledge from implicit representations to various stages of explicit representation which allow progressively more flexibility. In \citet{Karmiloff-Smith1986} she describes some particularly interesting evidence: children fail to balance an oddly shaped block when asked to do so, but successfully balance it when asked to build a house, because their procedural knowledge is more sophisticated than their explicit knowledge. \par 
This pattern of procedural knowledge preceding more explicit or object-like knowledge is supported by a much broader literature, from the gesture results of \citet{Goldin-Meadow1993} that I referenced above, to work suggesting that we progress from understanding mathematical concepts as processses to understanding them as objects \citep{Dubinsky1991, Hazzan1999}. If we do not already have good representations for a concept, we must create them by slow, procedural learning and reorganization, before we can begin to reason flexibly with that concept. However, it's worth noting that the process is bidirectional --- procedural knowledge supports conceptual understanding, but conceptual knowledge also helps improve procedural performance \citep{Rittle-Johnson2001}. Furthermore, procedural learning can occasionally be misleading. Getting too caught up in procedures that only work in simple cases can actually inhibit conceptual understanding \citep{McNeil2005}. There are sometimes trade-offs to transfer (see section \ref{fast_slow_interactions}). \par 
Is a separate rerepresentation process necessary to explain these phenomena? It is difficult to determine. Often rapid transitions and rule-like behavior can be fully captured within neural networks or statistical learning models more generally \citep[e.g.][]{McClelland1999, McClelland2002, Schapiro2009, Aslin2012}. While \citet{Karmiloff-Smith1992} argues that these mechanisms can not explain U-shaped developmental trajectories, since error-correcting learning will not apply where there are no errors, she ignores the potential effects of all the other learning that children are simultaneously doing on other related tasks. Furthermore, some compression emerges naturally from the learning dynamics of gradient descent \citep{Tishby2015, Shwartz-Ziv2017, Achille2017}, although there has been some debate about the generality of these observations \citep[e.g.][]{Saxe2018a}. It is difficult to rule out the possibility that complex developmental trajectories children exhibit can be explained by the combination of compression and the effects of error-driven learning on other related tasks. \par
\textbf{Summary:} We accumulate knowledge throughout the course of our lives. Some of this knowledge is implicit in the statistics of the world around us, while some is culturally constructed and transmitted. The quality of our knowledge representations can be improved by connecting to other knowledge, or by grounding in perceptual-motor understanding. Once we have acquired sufficiently high-quality knowledge representations, we are often able to transfer when we encounter a new task, and thereby learn more effectively than we could from \textit{tabula rasa}. This improvement in learning manifests as both more efficient learning and better generalization. \par

\subsection{Fast}
In this section I will discuss the systems that contribute to fast learning and transfer. It's important to note that these are systems that can be applied rapidly, but are not necessarily learned rapidly. Indeed, many of these reasoning systems are themselves culturally constructed, and must thus be learned over development. \par
\textbf{Hippocampal:} From the complementary learning systems perspective, the hippocampus serves as a fast learning system which can store an essentially unlimited number of distinct experiences while minimizing interference, i.e. as a nonparametric learning sytem \citep{Kumaran2016}. This feature makes the hippocampus an excellent sub-system for learning from a small amount of data, because it can store a few experiences and allow them to be retrieved at a later time. The hippocampus can also reduce catastrophic interference \citep{McCloskey1989} with knowledge gleaned from prior experiences, by allowing interleaving of these prior experiences in learning to integrate new knowledge with old \citep{McClelland1995}. This can potentially occur in various usefully-biased ways \citep{Kumaran2016, McClelland2020}. There may even be interesting computations performed within the hippocampus to support certain types of rapid generalization \citep{Kumaran2012}. \par 
\textbf{Interactive learning \& hypothesis testing:} Humans are able to use our stored experiences to rapidly learn. For example, we often behave as though we are formulating and testing hypotheses, even from a very young age \citep{Sobel2004, Gopnik2014}. We can even take advantage of these hypotheses in order to actively acquire information from the world that is most useful for us \citep[e.g.][]{Markant2014a}. By using our prior knowledge of the world to help interpret new experiences, we are able to make extremely fast inferences about how to understand a new situation. \par
\textbf{Education \& learned flexibility:} However, our ability to reason rapidly is not solely learned on our own. Indeed, an explicit focus of education is preparing us for future learning \citep{Bransford1999}, and flexibility \citep[e.g.][]{Richland2012}. That is, we are taught to learn increasingly rapidly as education goes on. Elementary school children require months or years of rote practice to learn arithmetic, but college mathematics students are expected to hear a theorem once and then immediately apply it. As noted above, we may not be perfect at these faster learning tasks \citep[e.g.][]{Hazzan1999}. However, adults are much better at them than children, and mathematics graduate students are much better than undergraduates \citep{Weber2001}. The ability to learn and apply new knowledge rapidly develops as we do. Furthermore, as we grow up we also grow better at explaining our actions, and adapting to instructions \citep[e.g.][]{Doebel2015}. Over the course of development, we practice many types of flexibility. \par 
\textbf{Explanations and demonstrations:} Both in education and outside of it, explanations are a key way we learn about the world, because they can succinctly convey rich structure \citep[e.g.][]{Keil2006, Lombrozo2006}. Importantly, explanations can be succinct because they exploit our prior knowledge in order to convey only the information that needs to be clarified (i.e. they follow the pragmatic principle of not being overinformative, expressed by \citet{Grice1975} for communication more generally). We not only learn from hearing explanations, but also from producing them \citep{Chi1989, Chi1994}, and so it is a common pedagogical principle to ask students to generate explanations. Explanations form a key tool for learning. \par
Similarly, we can also learn quite rapidly from demonstrations, which provide another succinct means for conveying important structure in the world. This learning requires applying our prior knowledge to infer what should and should not be generalized, but fortunately we can do so even from a young age \citep[e.g.][]{VanDamme2002}. Both explanations and demonstrations provide powerful tools for learning rapidly. We develop the ability to exploit these tools early in life \citep{Carpenter2005}. \par 
\textbf{Analogical and relational reasoning, and abstraction:} Some researchers have argued that analogical transfer and abstraction form crucial components of our learning \citep[e.g.][]{Gentner2003, Lakoff1980, Gentner2017}. These accounts often focus on fast, explicit transfer of the form explored by \citet{Gick1980}, for example. The structure-mapping algorithm \citep{Falkenhainer1989} proposed for analogical reasoning is based on explicitly searching over possible isomorphisms, which tends to be infeasible in practice. However, there may be ways to implement it efficiently enough that it could be considered for complex cognitive models \citep{Forbus2017}, and in past work I suggested that implicit learning could provide heuristics to dramatically speed up this search \citep{Lampinen2017a}. It's also reasonable to expect this ability to be related to education and other individual factors, since its been observed that features such as fluid intelligence may interact with the type of scaffolding provided to affect explicit analogical transfer \citep{Kubricht2017}. Indeed, older students are often better at transferring knowledge than younger ones \citep[e.g.][]{Chen1999}. These observations are suggestive of this type of transfer being a learned skill, rather than a cognitive primitive. \par
Much of the work on analogical and relational reasoning also focuses on the benefits of comparing multiple examples, which can lead to more reliable induction of abstractions or schemas and better transfer \citep{Gick1980, Gentner2017}. It has even been suggested that this comparison is a key way to understand the benefits of grounding \citep{Jamrozik2016}. However, in some past work, I found that seeing two different presentations of a mathematical concept lead to better learning overall than seeing either individually, but did not lead to significantly better abstraction of formal principles \citep{Lampinen2017b}. Thus it remains important to ask when reasoning about multiple examples leads to abstraction, and when it does not. One feature that is often important is explicitly considering the relationship between examples \citep{Gentner2017}, but it's likely that the quality of understanding of the examples matters as well. \par 
\textbf{Compositionality:} Many researchers have argued that cognition must rely on strictly compositional representations in order to exhibit the systematic and productive generalization that humans are capable of \citep[e.g.][]{Fodor2001, Fodor2008lot2, Lake2017}. However, an alternative approach has suggested that structured generalization can emerge from the structure of learning experience, without needing to be built into the model \citep{McClelland2010, Hansen2017}. This debate is complicated, and we will not try to resolve it. However, we will note that it can be difficult to define ``compositional representations'' in a way that offers an actual constraint on the model class \citep[c.f.][]{Zadrozny1992}. Furthermore, there have been exciting observations that altering aspects of the training regime, rather than the architecture itself, may allow more compositional generalization \citep{Hill2019a}. We will generally take the perspective that compositionality may be an emergent feature of learning, rather than a cognitive primitive. We will return to this discussion in greater detail in Section \ref{sec:introduction:on_compositional} below, as well as in Chapter \ref{chapter:conclusions}.\par
\textbf{Consciousness and explicit reasoning:} Consciousness is a slippery topic, but unfortunately must be discussed, as it underlies many of the fast-learning systems above. Most of these rely on our ability to explicitly reason about concepts once we have sufficiently high-quality representations of them. The global workspace theory of consciousness \citep{Baars2005, Dehaene2017} is closely aligned to this perspective. It states that conscious knowledge is precisely that which is globally accessible, and therefore with which we are most flexible. (C.f. also the higher-order-thought theory of consciousness \citep{Rosenthal1990}). This perspective concords to some degree with the perspectives of \citet{Karmiloff-Smith1986}, who argued that once we re-represent knowledge to be explicit, we can use it more flexibly. \par
However, \citet{Karmiloff-Smith1986} also argued that there are different levels of explicit representation of knowledge, and indeed some consciousness researchers have proposed more graded transitions from implicit to explicit \citep[e.g.][]{Cleeremans2002}. Computational models of graded transistions have been proposed, where explicit knowledge is essentially learned by a separate system which reasons over implicitly learned representations \citep{Cleeremans2014}. This fits more with the work reviewed above showing a graded transition to explicit knowledge, built upon implicit understanding grounded in percetual-motor features \citep[e.g.][]{Goldin-Meadow1993}, procedures \citep[e.g.][]{Hazzan1999}, or more basic concepts \citep{Wilensky1991, Patel2018}. It is also aligned with the perspective that structure should emerge, as above. Thus, when thinking about the relationship between implicit and explicit knowledge is unavoidable, we will take the perspective that explicit reasoning is built upon implicitly learned representations. \par 
Another place where explicit reasoning plays a role is in helping us to understand and learn from our own mistakes. For example, \emph{deliberate practice} (targeting specific aspects of performance for improvement) has been claimed to be key to developing expertise \citep{Ericsson1993, Ericsson2017}. Yet to engage in deliberate practice requires meta-knowledge about what needs to be improved, and indeed this abstract understanding is an important component of expert knowledge \citep{Feltovich2012}. \par
\textbf{Summary:} We possess fast learning systems that are engineered by evolution, such as the hippocampus, and ones that are culturally transmitted, such as our ability to follow instructions in our native language(s). Together, they allow us to infer a great deal from little information, by leveraging our slowly-accumulated prior knowledge. They also allow us to flexibly adapt our behavior through practiced algorithms like following instructions. \par 

\subsection{Interactions between fast and slow learning systems} \label{fast_slow_interactions}
I claim that most transfer arises as a synergy between different kinds of learning across different timescales. In particular, the knowledge we have accumulated over our lifetimes (part of which has been accumulated by our cultures over millenia) allows us to constrain the hypothesis space for new learning, so that we can make accurate inferences from a few examples in a new situation. \par 
From my perspective, this slowly-learned knowledge of the world can take multiple forms. It can occur in the mapping from inputs to our awareness, for example in visual cortex neurons which adapt to the regularities encountered over development \citep{Barlow1975}. However, it can also occur in the systems that implement higher-level and more rapid computations.\footnote{Clearly certain kinds of knowledge will lend themselves more easily to being learned in development and others will lend themselves more easily to being culturally transmitted or even learned by evolution. For the most part, I will generally assume that this knowledge emerges from experience or is culturally conveyed rather than being built in \citep{Hansen2017}, but my conclusions will mostly be agnostic to the origins of any particular piece of knowledge.} For example, the results on transfer in artificial grammar learning described above \citep{Tunney2001} or the results on transfer of simple harmonic oscillator strategies \citep{Day2011} show that humans are able to transfer knowledge at the level of structures or algorithms. The limits of this transfer are as yet unclear, as are the time-scales over which different kinds of transfer can occur.\par  
\textbf{Limitations \& tradeoffs:} Of course, there are trade-offs to relying on transfer and prior knowledge. When new tasks are not well aligned with our prior knowledge, relying on prior knowledge can actually interfere with learning. For example, this observation is one piece of the argument that we made \citep{Lampinen2017b} to explain the results of \citet{Kaminski2008}. This is an illustration of the broader phenomenon of negative transfer --- interference effects produced by transferring between non-isomorphic domains.\footnote{In fact the tasks in \citet{Kaminski2008} \emph{are} isomorphic at an abstract level --- part of our argument is that they are not isomorphic at the level at which participants actually engage with them.} A wide variety of studies have observed phenomena like this \citep[e.g.][]{Luchins1942, Landrum2005}. Our prior knowledge can be detrimental in situations that are very different from those we have encountered previously. \par
This brings us back to the broader point raised by \citet{Detterman1993} above, that humans are often unable to efficiently or flexibly transfer knowledge to new situations. Instead, this flexibility must be a goal of education \citep{Bransford1999}, and learning what is transferable may require developmental time \citep{Lampinen2017a} if the quality of the representations is insufficient to support faster transfer. We can be efficient and flexible when our prior learning has set us up to be. We are not always. We can be mislead by mismatches between the past and the present, or we can simply fail to find the correct analogy.\par

\section{Steps towards flexibility in deep learning}

A great deal of recent work in machine learning can be interpreted as attempts to engineer machine-learning systems with greater flexibility. In particular, much of this work has tried to allow them to learn from fewer examples, or generalize better to data that weren't seen during training. This work typically follows one of two approaches. The first, multi-task learning, focuses on learning multiple tasks with a single model, in the hope that the additional constraints on the model's representations will cause it to learn more rapidly or generalize better. The second approach, meta-learning, focuses on learning how to learn tasks, in the hopes of learning much faster and generalizing better from extremely small samples. I suggest that multi-task learning therefore relates to the ``slow'' transfer processes I discussed above, and meta-learning relates to the ``fast'' ones. In this section, I will give a brief overview of both these literatures, after some comments on generalization. \par

\subsection{On generalization in deep learning}
Before diving into the recent developments in flexibility, it is worth reviewing what we know about generalization in deep learning. Many problems of flexibility can be seen from another perspective as a problem of generalization. Ultimately the problem of cognition can be seen as a control loop from multimodal stimuli to responses \citep{Cisek1999,Cisek2019}. If we think of our lives as an unending sequence of inputs, influenced partly by our actions, behaving appropriately in any new situation we encounter can be seen as a generalization problem --- the question is simply whether we can recognize the relationships between a new situation and prior ones, and use those relationships to generalize appropriately.\par
As a clarifying example, a sufficiently good language prediction model should be able to do translation without ever having been explicitly trained on it, simply because it is a reasonable use of language. For example, if the model is conditioned on pairs of English and French sentences which are translations of each other, and is then provided with new English sentences as input, it should be likely to output appropriate French translations. Indeed, recent work demonstrates this striking generalization in state-of-the-art language models trained only to predict missing words \citep{Radford2019} --- see below for further discussion. Thus, what from one perspective can seem to be an altogether different task requiring adaptation, can from another perspective seem like basic generalization. From this perspective, different notions of flexibility are united under one general notion of generalization. \par
Moreover, many critiques of the inflexibility of deep learning highlight failures of generalization \citep[e.g.][]{Marcus2018}. To answer fundamental questions about the appropriateness of deep learning as a cognitive model, and to know whether we can trust AI systems to handle novel situations, it would be very beneficial to characterize the generalization capabilities of deep learning. \par 
Unfortunately, we understand relatively little about generalization. We know that a deep model which can memorize arbitrary labels for every ImageNet image nevertheless generalizes well on the real dataset \citep{Zhang2016}. How can this be true? We know that a relatively small network (by modern standards) can be Turing-complete \citep{Siegelman1992} --- that is, it can represent any computable function --- so how can it be that after training so many of these models compute close approximations of the right function? Classic machine learning generalization bounds relied on restrictions on how much a model could memorize \citep[e.g.][]{Vapnik1971}, so they cannot explain these results. In fact, these bounds are often anti-correlated with generalization performance, because more overparameterized deep models often generalize better! \par
There has been some recent progress on understanding these results, though that understanding is far from complete. With collaborators, we showed that (in a simpler deep linear model) the structure of the data, together with gradient descent and optimal stopping, effectively impose an inductive bias on the function computed \citep{Lampinen2019}. In particular, the most important dimensions of structure in the data are least contaminated by noise, and are learned earliest. Thus learning will capture most of the important structure in the data before corrupted and noisy signals are learned, ensuring that optimal stopping will yield good results. \par
Other lines of work have shown that aspects of the architecture may bias deep networks toward simpler regions of function space, which may therefore bias the networks towards solutions which generalize better \citep{Perez2019}. Still other work has shown that computing generalization based on weight norms or model compressibility can yield bounds that behave more appropriately with over-parameterization \citep{Neyshabur2018,Arora2018}. Thus, there are growing suggestions about how various features of our model architectures, algorithms, and datasets may contribute to generalization. However, there is not enough understanding yet to make strong claims about how to build models that generalize better. \par 
Thus, most progress in building more flexible deep learning models has been empirically driven. In the next sections, I will review some of that work. 

\subsection{Multi-task learning}

Multi-task learning is generally related to the ``slow'' learning systems I described in humans. The general idea is that relevant auxiliary tasks will serve as a useful inductive bias for the target task \citep{Caruana1997}. Typically, parameters are partially shared between the tasks, and these shared parameters are learned over long time-scales. This learning can be done either sequentially (where you use one task to pre-train the network for another), or simultaneously (where you learn multiple tasks at the same time or on alternating gradient steps). Auxiliary tasks need not be of the same type as the main task, for example reinforcement learning tasks can be supplemented with auxiliary supervised or unsupervised tasks like temporal autoencoding \citep[e.g.][]{Hermann2017}, and unsupervised tasks can be used to pre-train for supervised ones \citep[e.g.][]{Wu2018}. \par
\textbf{Pre-training:} One example of sequential multi-task learning is the extremely common practice of pre-training a network on some canonical task in order to use the representations of one of its hidden layers as a feature for learning some other related task. For example, pre-training on ImageNet \citep{Deng2009} is often seen as a useful way of constructing a general feature extractor for vision tasks \citep{Huh2016}, even for quite different transfer domains \citep[e.g.][]{Marmanis2016}. Even unsupervised pre-training can be helpful, since it helps identify the relevant axes of variation in the data, at least some of which are likely task relevant \citep{Erhan2010}. There is still active research on how to optimally pre-train features for various goals \citep[e.g.][]{Wu2018}. \par    
Pre-training is used on more than just vision tasks. In natural language processing (NLP) applications, the representations of words is often pre-trained on co-occurrence prediction tasks \citep[e.g.][]{Pennington2014}. The more general language modeling task (predicting words conditioned on past --- and possibly future --- words) can serve as unsupervised pretraining for many tasks \citep[e.g.][]{Radford2019}. Using larger sets of supervised language tasks as pre-training can be equally good, if not better \citep{Raffel2019}. In AlphaGo \citep{Silver2016}, the networks were pre-trained to predict expert go-players' moves, and then were tuned from that starting point using reinforcement learning. For other RL tasks, various pre-training approaches can be useful, such as training the agent to be able to reach diverse states \citep{Gregor2016}, or trying to learn adversarially discriminable skills \citep{Eysenbach2019}. The principle that pre-training a network can allow it to generalize better from smaller training sets appears to be quite general, although there is still some debate and ongoing research \citep[e.g]{He2018}. When it is possible to collect enough data on the target task, pre-training may no longer be necessary \citep[e.g.][]{Silver2017}. \par
There are also many remaining questions about how other features of training interact with transfer. Some researchers have argued that \emph{disentangled} representations --- loosely those that represent distinct features in distinct subspaces --- will be helpful for future tasks \citep{Higgins2018}, while others have argued that disentangled representations are hard to define, and that the evidence that they are useful is weak \citep{Locatello2019}. Some recent work suggests that some forms of regularization may actually harm transfer \citep{Kornblith2019}. One possible explanation for this result is that regularization is too compressive in the transfer setting --- it may compress away some real features that are useful for the transfer task, but not for the source task. However, the results might be different if the tasks were trained simultaneously (see below), rather than in a pre-training paradigm. Understanding when this interference occurs, and how to balance the generalization benefits of regularization with transfer benefits, will be an important area for future work. \par

\textbf{Curriculum learning:} A more general kind of pre-training is curriculum learning \citep{Bengio2009} --- the idea that training models on a structured progression of tasks can improve generalization. Pre-training on a single task is essentially a very simple curriculum. The idea of curricula began with \citet{Elman1993}, but was debated \citep[e.g.][]{Rohde1997}. However, subsequent work has demonstrated the importance of curricula both in toy settings \citep{Gulcehre2013} and in much more complicated tasks and models \citep[e.g.][]{Zaremba2014, Graves2016}. \par
Of course, having to devise a curriculum for each task you want to perform requires substantial human effort, which has led to work on automated ways of deriving curricula \citep{Graves2017}. Deriving curricula automatically is a difficult challenge --- while approaches like tracking progress on many tasks and training on the ones where performance is changing the most \citep{Baranes2013}, adversarially generating tasks of intermediate difficulty \citep{Florensa2018}, or learning curricula via meta-gradients \citep{Such2019} can work in simple domains, in more complex task settings the problem remains open. When the task consists of a two-player game, play between agents as they learn can form a natural curriculum --- the difficulty of the task increases precisely as the agent learns \citep{Silver2017, Jaderberg2019}. However, in more general non-competitive tasks, it is not clear how such an approach could help. \par 
In recent work, my collaborators and I explored a new approach to automated curriculum generation for goal-conditioned reinforcement learning \citep{Racaniere2019}. We combined several ideas from the previous work in novel ways, in order to scale these approaches up to more complex tasks, and tackled novel challenges, such as automated curriculum generation in environments where the possible tasks vary between episodes. We also highlighted a challenge to naive approaches --- as the complexity of the task space grows, it becomes very inefficient to explore task space at random, or by uniformly increasing difficulty. Most tasks in a complex environment will not be useful for the ultimate goal; for example, teaching a self-driving car to do a flip might be difficult, but it would not help with most tasks we want the car to perform. Human curricula are designed to efficiently lead learners to the desired competency, and we drew inspiration from this to propose a novel technique for pulling curricula towards a desired task distribution. These results represent a substantial step towards automated curriculum derivation in environments closer to the complexity of the real world. However, as we highlight in the paper, in more complex tasks it can be difficult to generate a curriculum without auxiliary information about the environmental structure. In these settings, the cultural knowledge behind human curricula, and the systems of mutually supporting tasks we have developed, will likely be necessary for achieving human-level intelligence. \par 

\textbf{Continual learning \& avoiding interference:} Curriculum learning raises another issue --- what if you want your model to perform well at several tasks? If you switch to training on a new task, it may catastrophically interfere with your ability to perform a previous task \citep{McCloskey1989}. The complementary learning systems perspective \citep{McClelland1995, Kumaran2016} was intended in part to address this issue. Some of our new work in this area has shown that (at least under some circumstances) replay can be quite efficient --- old items need only be replayed proportionally to how related they are to the new, interfering items \citep{McClelland2020}. \par
A number of other approaches have also been proposed recently, based on ideas like preserving parameters from updates proportionally to how important they are for old tasks \citep{Kirkpatrick2016, Zenke2017}, or learning tasks separately but distilling the knowledge into a single network later \citep{Rusu2015}, or by trying to use unsupervised learning to find good representations and allocate new tasks so that they don't interfere \citep{Achille2018a, Rao2019}. Using HyperNetworks \citep{Ha2016} to specify task specific parameters may be helpful, as was observed contemporaneously in the preprint versions of \citet{Lampinen2019a} and \citet{Oswald2020}. There are also some meta-learning and memory-based approaches discussed in section \ref{meta_learning_sec} that can ameliorate this problem. Thus there are a variety of approaches that can address catastrophic interference, while still maintaining the benefits of curriculum learning. See \citet{Parisi2019} for a thorough recent review. \par
There is an additional challenge --- much work on continual learning assumes that the boundaries between tasks are known, and that task identities are known. However, relaxing one or both of these assumptions might be more realistic \citep{Ven2018}. Some recent work has attempted to address this issue by developing algorithms that infer the tasks and task transitions from a continuous stream of data \citep{Nagabandi2019}. \par
However, beyond simply avoiding interference, a major hope is that prior knowledge will help with learning of a new task. Ideally, prior knowledge would be useful even in superficially dissimilar domains, if the underlying structure is similar. While most curriculum learning work at present still samples the curriculum from a very narrow set of tasks, such as navigation goals of varying difficulty \citep{Florensa2018}, humans seem to be able to leverage analogies from much more disparate domains, such as using the flow of water to understand the flow of heat \citep[see][]{Falkenhainer1989}. It remains a challenge for machine-learning systems to learn different types of knowledge from the variety of tasks that humans experience, in such a way that prior learning supports new learning rather than interfering with it \citep{Mitchell2018}. I will provide some of my perspectives on the potential for positive transfer in Chapter \ref{chapter:timescales}. \par 

\textbf{Simultaneous multi-task learning:} Many multi-task learning approaches train on the tasks simultaneously rather than sequentially. For example, simultaneously training a natural language translation system to do image captioning in the target language improves its translation performance \citep{Luong2016}. Even training it to translate between multiple language pairs is beneficial \citep{Dong2015}. This approach can even lead to zero-shot generalization to translation between language pairs never seen together in training \citep{Johnson2016a, Platanios2017}. Training a shared model on many natural language tasks can substantially improve its generalizability \citep{Raffel2019}. For this reason, recent natural language benchmarks have focused on broad sets of challenging tasks rather than single-task evaluation \citep{Wang2019,Wang2019b}.\par 
The idea of multi-task learning has proven even more critical in Reinforcement Learning (RL), where auxiliary tasks have been suggested as a key approach to overcoming the problem of reward sparsity \citep[e.g.][]{LeCun2016}. Auxiliary tasks have been used in a variety of RL settings, for example in grounded language learning \citep{Hermann2017} or game playing \citep{OpenAI2019, Vinyals2019}. \par
The broad observation that deep networks will learn representations that represent shared structure in the tasks they perform, and can exploit this shared structure to generalize better, is not new. It was observed at least as long ago as \citet{Hinton1986}, and has continued to intrigue researchers since \citep[e.g.][]{Lampinen2017a}. In particular, it is a key feature that separates deep networks from simpler statistical learning architectures, and allows them to uncover and exploit deep structure in the world \citep{Rogers2008}. This structure extraction may support some of the most interesting kinds of transfer that deep networks demonstrate, and may have provided substantial benefits in the various projects reviewed above. \par 

\textbf{Learning which parameters to share:} Most the work above simply fixes architectures with pre-specified shared and unshared weights. However, there are other approaches that attempt to learn which weights to share. The work of \citet{Achille2018a}, referenced above, is one example of learning what to share. Other authors have considered using evolutionary algorithms to decide which subsets of modules should be shared \citep{Fernando2017}. Some other approaches can also be seen from this perspective, for example choosing a sparse subset of modules for each forward pass \citep{Shazeer2017} or using a HyperNetwork to generate the weights for other networks \citep{Ha2016}. This latter work has led to a great deal of productive research in domains ranging from language \citep{Platanios2017} to meta-learning of visual classifiers \citep{Garnelo2018, Li2019a}. Both HyperNetworks and using subsets of modules can be seen as a way of learning which weights should be shared or separate between different ``sub-tasks'' of a task. Some of the work discussed in section \ref{meta_learning_sec} can also be viewed as learning what parameters should be shared and which should be separate. While in principle a fully-shared network could learn which parameters to share just by gradient descent, in practice a more structured approach to this problem can be useful. \par 

\textbf{Tasks as an input feature:} The most flexible approach to multi-task learning involves simply providing task representations as an input to the model and letting everything be learned, rather than pre-specififying anything about how computations should be shared and separated. This approach has the benefit that the system can potentially learn to generalize to novel tasks. Research on general value functions in reinforcement learning \citep{Sutton2011} --- value functions which take a task specification as an input --- provide one inspiration for this approach. Task-conditioned RL approaches have exhibited success in certain cases, for example generalizing to unseen natural-language task specifications \citep{Hermann2017}. With increasing dataset size, this approach is becoming more feasible in complex natural langauge processing tasks \citep{Raffel2019}. \par

\textbf{Reinforcement learning:} Reinforcement learning (RL) has a long history within neuroscience \& cognitive science \citep{Odoherty2003,Niv2009,Dabney2020}, and artificial intelligence \citep{Sutton2017}. RL tackles the fundamental problem of learning in tasks where an agents actions affect the environment it is in. This interaction means that it is impossible for the agent to observe and learn from all counterfactual possibilities, unlike in supervised learning where a model may update its predictions for every alternative label. It also means that supervision can be harder to provide, because the supervisory signal must take into account the actions of the agent. Both of these factors make RL settings more realistic for real world tasks. They may therefore provide an important bridge to more flexible models.\par 
Indeed, some of the major recent successes of deep learning have been driven (at least in part) by reinforcement learning. For example, using RL to play Atari videogames \citep{Mnih2015} was a powerful demonstration of the ability of deep learning to perform complex tasks. More recently, RL-based models have achieved superhuman performance in complex games such as Go \citep{Silver2016,Silver2017}, and expert human performance in extremely complicated video games \citep{OpenAI2019,Vinyals2019}. \par
However, there remain difficult challenges to applying RL to complex problems, and it remains an area of active research. Applying genetic algorithms might increase data-efficiency in some cases \citep{Petroski2018}, and representing value as a distribution (instead of a point estimate) may improve performance \citep{Bellemare2017}. Clever replay schemes, such as relabeling data, may allow for more effective learning from a small amount of experience \citep{Andrychowicz2017}. Using memory lookups as a cue may help solve the long-term credit assignment problem \citep{Hung2019}. Still, RL algorithms are often unstable, and seemingly inconsequential changes like rescaling rewards may substantially alter results \citep{Henderson2018}. Making RL reliable on complex tasks remains a challenging goal. \par  

\textbf{Summary:} To summarize, curriculum and multi-task learning can help set up representations that allow deep learning models to generalize better from less data, or to learn tasks that would not otherwise be learnable. They do so because the additional constraints imposed on the network's representations through auxiliary tasks can help the network to uncover the true underlying structure in the environment. While multi-task learning and curriculum approaches do not support all the kinds of flexibility that humans demonstrate, they are a key piece of the puzzle of how deep networks can learn faster and generalize better, and may be one important feature that separates humans from our best current models. \par

\subsection{Meta-learning \& related approaches} \label{meta_learning_sec}

The fundamental insight of meta-learning is that there is a continuum between data and tasks. We can interpret each (input, target) tuple as a simple task, and we can interpret subsets of a dataset as sub-tasks, for example all the dog images contained within ImageNet form a semantically distinguishable sub-task of the overall task. Analogously, we can often interpret a single task as being just a point in a larger space of tasks. Most meta-learning models exploit these insights by having the architecture adapt to a given task within its activations instead of its weights, just as a CNN would adapt to the fact that its current input was a dog image rather than a tree image within its activations. Learning to adapt to new tasks in this way has been shown to allow for much more efficient learning on new tasks, and may be key to modeling human intelligence \citep{Hansen2017}. \par 
\textbf{Basic meta-learning:}
The basic meta-learning approach (representing a task within activations) has been applied to many settings. For example, it has been used to learn to classify new images based on single positive exemplars of each class \citep{Vinyals2016, Ravi2017}. It has also been used to teach recurrent networks to solve simple reinforcement learning problems \citep{Duan2016, Wang2016a, Stadie2018}. \par
There have been many variations on this approach. Some meta-learning work has exploited both slow and fast weights with some success \citep[e.g.][]{Munkhdalai2017}, taking one perspective on the dichotomy I proposed above. Many approaches have exploited other tricks that are broadly useful in machine learning. For example, some work has leveraged unlabelled examples along with a few labelled ones to yield better meta-learning results \citep[e.g.][]{Ren2018}. Other work has shown that attention-based models are useful \citep{Reed2017}, as in many other machine-learning applications \citep[e.g.][]{Vaswani, Gregor2015}. Many recent approaches have constructed task representations, and used these to classify new data points \citep{Li2019a,Ravichandran2019}. Finally, some work has shown that Bayesian inference may be a useful tool in these settings \citep{Burgess2016}. \par
\textbf{Optimization-based methods:} Another productive line of meta-learning work uses an inner optimization loop to tune the model to each task. This work largely follows from the work of \citet{Finn2017a} on Model-Agnostic Meta Learning (MAML), an approach based on optimizing the model so that it could adapt well to a new task in a few gradient steps. MAML has led to many follow-ups \citep[e.g.][]{Finn2017, Finn2018, Nichol2018}. At least in some settings, it may only be necessary to adapt a few task-specific parameters rather than the whole model \citep{Zintgraf2018} --- this approach may be thought of as a form of task representation. \par

\textbf{Memory-based:} There are a variety of meta-learning approaches that are based on a non-parametric memory, and generally some form of key-value attentional lookup over it. A general extension of the basic meta-learning approach to the case with memory is given in \citet{Santoro2016}. Other approaches are based on using memory only at testing, e.g. by tuning the network rapidly to perform well on similar examples, as proposed by \citet{Sprechmann2018}. \par 
More flexible variants have also been proposed. For example, the differentiable neural computer (DNC), proposed by \citet{Graves2016} is able to receive a graph structure as inputs, learns to store it in memory, and then learns to use that stored information to solve problems like computing shortest paths. That result can essentially be seen as meta-learning --- the architecture has the flexibility to learn and reason from new knowledge rapidly. However, this flexibility remains fundamentally within the computations done for a task which is slowly learned, and does not by itself allow the more general flexibility that humans exhibit. \par

\textbf{Language \& zero-shot performance:} There is some work that has explored related problems from a language-based perspective. For example, \citet{Larochelle2008} considered the general problem of behaving in accordance with language instructions as simply asking a model to adapt its response when conditioned on different ``instruction'' inputs. Since this approach does not require data on the new task, it is a form of zero-shot learning. Later work explored zero-shot classification based on only a natural language description of the target class \citep{Socher2013,Romera-Paredes2015, Xian2018}. Many of these used very simple language, e.g. a single word (the target class label), and used tricks such as combining prior classifiers based on their labels' word-vector-similarity to the target \citep[e.g.]{Norouzi2014, Changpinyo2016}. More recently, there has been a productive line of research in using language to compose network modules for question answering \citep{Andreas, Andreasa}, at least in toy domains. \par
There has also been some work on reinforcement learning systems that learn to follow natural language instructions in simple environments \citep{Hermann2017, Oh2017a}. In past work with collaborators, we showed that more realistic environments improved the generalization exhibited by these instruction-following systems \citep{Hill2019a}. Other work has shown that language can form a useful intermediate representation for a simple form of hierarchical reinforcement learning \citep{Jiang2019}. \par
More recently, \citet{Radford2019} showed that a language model trained on an extremely large corpus of curated websites actually acquires some meta-like abilities, for example the ability to translate languages ``zero-shot'' (i.e. when conditioned on examples of translation pairs and then given a new sentence, it produces a translation). It is also able to summarize, answer questions, etc. It presumably is able to accomplish these tasks because the corpus contains some translation pairs in context or summaries on a page, and so those tasks essentially compose a small part of the whole language modeling problem. It's worth noting that the performance on these tasks is rudimentary, and far from models that are trained for these tasks in a supervised way, but it is still an impressive demonstration of the power of prediction to extract deep latent structure in the world, and the power of broad enough training data distributions to teach flexibility. \par  

\textbf{Demonstrations:} The issue of learning from demonstrations has also been considered in the machine learning literature for some time, because of its potential applications in problems where we know what behavior we want, but not how to encode it. There the problem has generally been referred to as ``inverse reinforcement learning'' \citep{Ng2000}, i.e. the problem of inferring a value function from observed behavior. A large number of approaches to this problem have been proposed \citep[e.g.][]{Ng2000, Abbeel2004}. Recently, approaches combining meta-learning and deep learning have achieved some success. For example, \citet{Finn2016} present an algorithm which can learn to infer a reasonable approximation of the objective function from a single demonstration. This work shows that neural networks can meta-learn to learn from demonstrations. \par

\textbf{Relational and analogical reasoning:} There are a number of other approaches that attempt to explicitly build relational inductive biases into deep-learning architectures. Relation networks \citep{Santoro2017} build in this prior by having the architecture explicitly relate different parts of the input, and achieve better performance on answering relational questions about visual scenes. Architectures that directly allow for true relational binding can be beneficial for a variety of applications, especially natural language processing or symbolic reasoning \citep[e.g.][]{Smolensky1990, Smolensky2014, Huang2017}. Graph-structured architectures form a very natural way of representing few-shot learning problems \citep{Garcia2018}, and more generally graph-structured or other relational inductive biases have been suggested as a promising direction in deep learning \citep{Battaglia2018}. How these inductive biases benefit (or limit) learning is an important direction for future research. \par 
However, relational reasoning is constrained by training as well as the architecture. For example, choosing the negative examples that a network learns from to explicitly contrast relational hypotheses can help to yield more relational reasoning \citep{Hill2019}. Exploring how architecture and training interact to produce relational reasoning will be an important future direction. \par

\textbf{Abstraction:} There has also been some work on explicitly building abstraction capabilities into machine learning systems. For example, in reinforcement learning the idea of options \citep{Sutton1999} and hierarchical reinforcement learning more broadly \citep[e.g.][]{Botvinick2009} are essentially encapsulations of temporal abstraction, where a sequence of actions can be represented as a single higher-level action. For example, we can think of going to the office as a single action, rather than a sequence of many steps. Similar attempts have been made to allow deep learning models to share knowledge across tasks, with some success \citep[e.g.][]{Tessler2016}. Other approaches have tried to infer hierarchical task representations during meta-learning, for better generalization \citep{Yao2019}. \par 
There have also been a variety of attempts to combine deep learning methods with approaches based on programming. For example, Neural Programmer-Interpreters \citep{Reed2015} essentially endow a recurrent network with the ability to call sub-routines, and a stack of memory for these sub-routines to use. Applying these ideas to meta-learning problems has been reasonably successful, especially with carefully chosen algorithms for integrating knowledge across tasks \citep[e.g.]{Devlin2017}. Similar techniques have been applied to many domains, such as learning from demonstrations \citep[e.g.][]{Xu2017a}. Combining the old ideas of cognition as executing symbolic programs \citep{Newell1961} with the techniques of deep learning can yield improvements in flexibility. \par  
However, most of these methods are still not as universally flexible as humans. The number of abstractions is usually fixed, often abstractions cannot be composed from other abstractions, and abstractions are inflexible to other demands. For example, a system that has learned an option for walking to a goal will not necessarily be able to change to running to the goal without learning this option from scratch. Thus there are still limitations to these approaches at present. \par 

\textbf{Model-based reinforcement-learning:} Model-based RL methods provide a useful factorization of the RL task, that can allow the same model to be used for a new task with a different reward function \citep[e.g.][]{Laroche2017}. More flexible hybrid model-based methods, such as letting an agent learn to plan \citep{Tamar2017}, show potential promise. However, many of these suffer from stability issues, as prediction errors compound over rollouts \citep{Talvitie2014}. However, treating these rollouts as a potentially flawed imagination, and letting the model learn to interpret them can help \citep{Racaniere2017}, as can rolling out in latent space rather than in observations \citep{Gregor2019}. It seems likely that one component of flexibility will be learning models that can be reused for new purposes. However, it is as yet unclear what those models will look like, and whether they will need to have planning as an inductive bias at all. At least in some circumstances, planning-like behavior can emerge in a model-free architecture with an appropriate recurrent structure \cite{Guez2019}. \par
There has also been a substantial amount of work on the succesor representation, which is a hybrid between model-based and model-free methods that caches state transitions and values. The successor representation may serve as a useful compromise between the model-free and model-based methods in some cases, and there is some evidence that humans create successor-like representations on some simple tasks \citep{Momennejad2017}. However, the successor representation cannot adapt well if state transition probabilities change drastically, and other approaches such as task clustering must be adopted to accomodate these challenges while maintaing flexibility \citep{Madarasz2019}. \par
Furthermore, both model-based and successor-representation-based methods only handle replanning if given a new reward or value function. They thus offload a substantial part of the problem of adaptation to another system. Combining these methods with the methods I propose in later chapters might allow for a more complete solution to the problem of adaptation.\par

\textbf{Other work:} There is a variety of other work that has exploited different perspectives on meta-learning. Some of this work could be useful for thinking about flexibility. For example, \citet{Xu2018} proposed using meta-learning to adapt hyperparameters of reinforcement learning algorithms across tasks. Other work has attempted to meta-learn auxiliary tasks for transfer, based on improvement on target tasks \citep{Liu2019a}. Some work has even attempted to combine these, using meta-gradients to choose auxiliary tasks \citep{Veeriah2019}. Other research has shown that reframing continual learning as a meta-learning problem (of learning to learn without interference) can be effective \citep{Velez2017}, at least in simple settings. All of these approaches allow for better adaptation to new tasks or environments. \par
In addition, there has been some work showing that basic neural network models can adapt rapidly to new data that is consistent with prior knowledge, simply by optimizing weights specific to that data while freezing the remaining weights of the network \citep{Rumelhart1993}. This approach only works if there are weights that are specific to the new item(s), so it is not a general kind of flexibility. Nevertheless, this approach has been applied to understand human semantic cognition \citep{Rogers2004}. More recently, I applied it to one-shot and few-shot learning of words in a language-modeling task \citep{Lampinen2018a}. Thus under some circumstances adapting item-specific parameters can yield a certain kind of flexibility, even at the scale of large machine learning tasks. This observation provides further evidence for the general point that information which is consistent with prior knowledge can be rapidly integrated \citep{McClelland2013, McClelland2020}. \par 

\textbf{Unintentional ``flexibility'':} There has also been some interesting work on flexibility that emerges accidentally under standard training of deep learning systems. In particular, adversarial examples \citep{SzegedyAdv} are cases when adding a very small perturbation to an input can radically alter the network's output. This drastically altered behavior is a kind of flexibility, but it is not the desired kind. Instead, these appear to be evidence for the fact that deep networks are inherently chaotic systems, which can respond in surprisingly sensitive ways to their inputs. However, it's worth noting that humans can be susceptible to more extreme adversarial perturbations derived on deep networks \citep{Elsayed2018}, and that many perturbations are human-interpretable even if we would not make the same mistake \citep{Zhou2019}.\footnote{Although this latter claim has been debated \citep{Dujmovic2020}.} Furthermore, adversarial examples can be exploited, e.g. for better training \citep{Goodfellow2015}. \par 
\citet{Elsayed} demonstrated an even more interesting type of unintentional flexibility: deep networks can be ``reprogrammed'' by an input to solve a different task.\footnote{Although not one completely unrelated to the main task the network was trained on --- the network was trained on a vision task, and their reprogrammed tasks were just other vision tasks. It would be interesting to explore the limits of this ``reprogrammability.''} This reprogrammability is a step closer to the flexibility that humans have, but the ``reprogramming'' inputs have to be derived via an optimization process, and tend to be uninterpretable. Thus there is no systematicity in this flexibility. However, my interpretation of these results is that they are encouraging evidence that these models have the capacity to be extremely flexible under appropriate conditions. All that is needed is to train the models to be flexible systematically. \par  

\textbf{Meta-learning AI itself:} \citet{Clune2019} argues that we should meta-learn all aspects of the AI engineering process. In particular, he suggests that we should meta-learn the architectures, algorithms, and the tasks that we use to train our AI systems. This approach is indeed an exciting direction for future work, and may ultimately prove fruitful, but in these settings it is more complicated to determine what the over-arching reward or loss should be, and how to represent the features of learning themselves. Recent work has become to hint at potential solutions to these problems, but it will be some time before we can validate these techniques on tasks of the scale at which more mature architectures and algorithms are evaluated. \par 

\textbf{Summary:} To summarize, meta-learning has made progress on several fronts. It is starting to solve the small data problem, by allowing networks to learn efficiently from a small number of examples \citep[e.g.][]{Wang2016a}, although they require a large number of training tasks in order to do so. At the other extreme of very large data, deep networks can generalize to some extent to tasks which are only hinted at by the trained task distribution \citep[e.g.][]{Radford2019}. It has been suggested that meta-learning the AI architectures, algorithms, and tasks themselves may be the approach to creating artificial general intelligence in the futuer. However, there are a number of key features of human flexibility that remain unexplained by current approaches. In the next section, I will relate transfer and flexibility in humans and deep networks, and discuss the features that are still missing. \par 

\section{Relating flexibility in humans and neural networks}

The encouraging progress on multi-task and meta-learning in recent years suggests that cognitive models exploiting these techniques may help explain human flexibility and transfer \citep{Hansen2017}. In this section, I will attempt to relate the aspects of human and network flexibility that I outlined in the previous sections. \par 
First, it generally seems that the division between fast and slow transfer is applicable both to the machine learning literature (meta-learning vs. multi-task) and to human transfer, as we highlighted above \citep[and in prior work, namely][]{Lampinen2017a}. Following complementary learning systems, I suggest that broadly our slow learning of structure in the world happens over the course of developmental time in a multi-task fashion. Algorithmically, I suggest that this occurs because networks in our brain come to represent and exploit structural similarities across the many tasks we experience. We also explicitly practice using these slowly-learned representations to support rapid transfer \& learning, in educational settings as well as in everyday experience more broadly. Thus I think that cognitive models should employ both slowly-learned shared representations, and a system that allows for rapid and flexible reasoning over them. I argue that incorporating both aspects will be key to modeling the full range of human flexibility --- one of my main goals in this dissertation will be to construct a model which exhibits these characteristics. \par 
It is also important to note the differences between the systematic, structured training that humans encounter in our systematic development and culturally-constructed educational systems, and the unstructured, IID training that deep learning models canonically receive \citep{Smith2017}. While curriculum learning addresses some of the sequential learning in development and education, and meta-learning addresses part of the learning-to-learn aspect, the full training on flexibility is generally missing. For example, humans learn a great deal from explaining as well as simply doing, yet we rarely train our machine learning models to explain their actions. Given the sensitity to training data that both humans and neural networks display, we cannot expect deep learning models to capture human behavior completely under drastically different learning regimes. It is important to develop richer, more structured educational paradigms for neural networks, both in order to use them as models of human intelligence, and to develop more human-like artificial intelligence systems. \par
In summary, I believe that deep learning systems remain promising cognitive models. They have successfully modeled a wide range of phenomena ranging from low-level neural activity to cognition. Furthermore, they are compelling because they are some of the only systems to successfully achieve human performance at difficult tasks like visual object recognition or playing go. They even have some inherent (if often unstructured and unintentional) flexibility, as indicated by adversarial examples and reprogramming, and more recent methods like meta-learning have given them more systematic flexibility. \par
I suggest that humans are similarly flexible. I suggest that the key to our flexibility is that we learn over the course of development and education to exploit our flexibility in systematic ways, in order to be adaptable in new tasks and situations --- just as a meta-learning system learns over many tasks how to learn rapidly on a new one. This flexibility does not necessarily need to be an explicit target of the learning procedure, e.g. the results of \citet{Radford2019}, discussed above, show that training a language model on a large enough text distribution gives some generalization to related tasks like translation. However, that network required far more training than could be assumed for humans, and still lacked some of the flexibility that humans have. I suggest that its weaknesses are due to its lack of multiple tasks to constrain the representations, the lack of an architecture explicitly designed to allow synergies between fast and slow learning, and the lack of the systematic, structured training at the scale that humans experience. However, there are alternative perspectives. We will consider one of the most frequent ones in the next section. \par 

\subsection{On compositional symbol manipulation} \label{sec:introduction:on_compositional}

One of the most frequent criticisms of deep learning is that it lacks the compositional symbol manipulation ability that humans possess \citep[][see also above]{Fodor1988, Lake2016,Lake2017,Marcus2018}. This argument has inspired a variety of works that incorporate symbol-like processes into the representations of deep models \citep[e.g.][]{Andreas2017,Mao2019}. These works explicitly constrain the representations of the models to use language-based representations, or symbolic (perhaps probabilistic) programs. \par 
Before we discuss the merits of these approaches, it's worth considering the motivation. The notion of compositional cognitive representations was introduced by \citet{Fodor1988}, motivated by the importance of compositionality as an axiom in linguistics. However, within linguistics there is a growing recognition that it may be necessary to discard strict compositionality to properly understand semantics \citep{Goldberg2015,Potts2019}; ironically, this change is due in part to the success of deep learning models in natural language processing domains. In fact, theoretical work shows that the strict notion of compositionality does not even constrain semantics --- any semantics can be rewritten to be compositional \citep{Zadrozny1992}. Furthermore, even \citet{Fodor2001} argued that language is not compositional (although he maintained that thought is).\footnote{It is not even clear if syntax is safe as a computational principle of language processing, as recent work has shown that language-selective brain regions respond mostly to local word transition probabilities, rather than syntax \emph{per se} \citep{Mollica2020}.} Given that the compositionality of language motivated the arguments for compositionality in cognition, what should we think if language is not strictly compositional? To elaborate on this point, consider the argument of \citet{Potts2019}:
\begin{quote}
``The usual story is that compositionality is crucial to our ability to produce and understand creative new combinations of linguistic units, because it offers guarantees about the systematicity and predictability of new units. However, these observations alone do not imply compositionality. The interpretation of a given phrase could be systematic, predictable, and also determined in part by global properties of the utterance, the speaker, the discourse situation, and so forth. And, indeed, it seems to me that our everyday experiences with language are in keeping with this.''
\end{quote}
It seems to me that similar reasoning applies to the compositionality of representations in cognitive models. \par
Despite these voices, the debate over compositionality persists, both within linguistics and cognitive science. I think that part of the reason is that compositionality can be challenging to define. The arguments above apply to a standard linguistic definition: ``the meaning of a complex expression is determined by its structure and the meanings of its constituents'' \citep{sep-compositionality} --- that is, that the meaning of a complex expression cannot be influenced from context outside the expression. Part of the conceptual challenge is that researchers sometimes equivocate between this strong notion of compositionality and weaker notions, e.g. that ``expressions have internal structure.'' However, to the extent that it is argued that compositionality is a feature deep learning models lack, the definition used cannot be a weaker definition, since the representations of these models acquire internal structure insofar as it is afforded by the training data \citep[e.g.][]{Mikolov2013}. \par
Indeed, the state-of-the-art systems on complex natural language tasks do not use any symbolic or compositional inductive biases \citep[e.g.][]{Radford2019,Raffel2019}. By contrast, the works that incorporate language or programs into the model representations are typically demonstrated on small toy experiments where the world decomposes nicely into simple elements \citep{Andreas2017,Mao2019}. Even within these carefully constructed environments, when symbolic inductive biases are compared to more end-to-end approaches, the latter often prove superior. For example, using language as a latent bottleneck on a toy visual meta-learning task improves task performance compared to not using language at all \citep{Andreas2017}, but using language as just an auxiliary signal performs even better \citep{Mu2019}. At least in that setting, language is a useful learning signal, but a harmful constraint. It is often the case that building in inductive biases and domain specific knowledge has proven detrimental in the long run --- \citet{Sutton2019} calls this observation the ``bitter lesson that can be read from 70 years of AI research.'' \par
As another example, relatively unstructured deep networks sometimes outperform carefully engineered symbolic methods even in strictly symbolic domains like mathematics. \citet{Lample2019} recently showed that a deep transformer model outperforms Mathematica (and Maple and Matlab) at symbolic integration and solving differential equations \citep{Lample2019}. Given that Mathematica is an extremely sophisticated symbolic reasoning system, and mathematical reasoning is probably the most symbolic human task, this result is quite surprising. However, \citet{Davis2019} pointed out that the comparison is not truly balanced. That response raises some valid points about differences in the set of possibilities and solutions that the two approaches are considering. It is clear that we could not yet replace Mathematica with a deep network. However, it does make it more challenging to assert that deep networks cannot do compositional symbolic reasoning. \par
One objection raised by \citet{Davis2019} is that, while the deep network is 98.4\% accurate (at 1 beam) in the domain of problems given, compared to 84\% from Mathematica, Mathematica will never produce an incorrect answer. That is, Mathematica is either correct or it times out, whereas the output of the model is not deterministically informative about whether the answer is correct. This objection is valid, up to a point, and is one of the reasons that deep models will not fully replace Mathematica anytime soon. However, when it comes to human reasoning, this point becomes more challenging. When humans make a compositionally-valid interpretation 80\% or 95\% of the time, it is presented as evidence of their compositional symbolic skill \citep{Lake2019a}, yet when a deep model achieves 98.4\% accuracy on much more difficult problems, it is depicted as a failure of the model class to exhibit compositional symbolic reasoning. Because it is difficult to know how to attribute mistakes in either humans or models, and it is difficult to match the scale and experience of humans in our contemporary models, it is not clear how to produce a fair evaluation of both that would decide the issue of whether symbol manipulation is a necessary ingredient of cognition. \par 
It seems intuitive that symbol-like systems and particular compositions of representations would be helpful for particular types of reasoning. However, maintaining a strictly symbolic or compositional representation requires that the decomposition be imposed a priori. If there are many possible ways to decompose our knowledge, it might be more useful to flexibly decompose and represent knowledge according to the task at hand. This trade-off might be one reason that compositional inductive biases could be helpful in toy cases, but harmful on complex real-world data.\par
Instead, one might hope for a model that can construct appropriate decompositions, and could therefore allow for any inference in an appropriate context. To do so, the model would need to avoid considering the space of all possible decompositions, because one of the main issues with using strictly compositional representations (and symbolic reasoning more generally) is the exponential number of possible inferences in complex tasks. However, one key success of deep learning models is to find extremely good direct approximations that reduce (or entirely avoid) searching. Thus, one might hope that compositional structure might emerge in the representations and behavior of a deep learning model, insofar as it is relevant and useful.\par 

\subsection{On scale \& emergence}

Thus, an alternative possibility is that compositionality and symbolic-rule-like reasoning may be an emergent feature of learning, rather than a learning mechanism \citep{McClelland1999, McClelland2002, McClelland2010, McClelland2010a}. Indeed, facility with symbolic mathematical reasoning emerges only after a great deal of experience in the domain \citep{Burger1986, McClelland2016}. Likewise, cross-cultural comparisons have shown that speakers of a language without number words are unable to express or remember exact quantities \citep{Frank2008}. Similarly, \citet{Gleitman1970} showed that only graduate students (and not undergraduates or high school graduates) exhibited a strong ability to understand three-word compound nouns, despite these being a relatively basic construction that should be easily parsed according to the syntactic rules of the language. It is important to consider the possibility that our ability to generalize in symbol-like ways is an emergent property of our lifetime of experience in cultural systems that emphasize formal reasoning. It is difficult for researchers to think outside of the cultural framework in which we have been educated, and to consider how differently we might reason if our education had been different. \par 
This obvservation leads to the possibility that our current deep learning training paradigms are just too unstructured, our tasks too simple, and our models too small for human-like flexibility to emerge. If we could immerse a deep network with as rich and recurrent an architecture (and as many parameters) as the brain in a rich lifetime of experience and education, would it be able to reason like a human? Changes in scale can often result in qualitatively different \emph{emergent} behavior --- ``more is different'' \citep{Anderson1972}. It has been suggested that this emergence might underlie many important aspects of human intelligence \citep{McClelland2010a}, from semantic cognition \citep{Rogers2008, Saxe2019} to consciousness \citep{Chalmers2006}. It is important not to underestimate the difference in scale between deep learning models and the human brain. While extremely large models may have billions of parameters \citep[e.g.][]{Radford2019}, the human brain has hundreds of trillions of synapses \citep{Drachman2005}, each of which is much more complex than a single weight from an artificial network. Furthermore, while modern meta-learning approaches may expose a machine learning system to many closely related tasks, they do not approach the years of experience in disparate domains that humans experience \citep[c.f.][]{Mitchell2018}. There is room left for emergence.\par 
Indeed, changes of scale have driven many of the recent successes of deep learning. The rise of deep convolutional neural networks in computer vision was driven in large part by increasing dataset size \citep{Deng2009}, combined with increasing computational power and efficient implementations of neural networks on GPUs \citep{Krizhevsky2012}. It is both intuitive, and supported by long-standing theoretical results \citep{Bartlett2002}, that generalization improves with increasing dataset size. As noted above, recent work in machine-learning theory has shown the less intuitive result that qualitatively different results may occur when optimizing deep neural networks with many parameters than shallower or smaller ones --- overparameterization can actually be beneficial \citep{Dauphin2014, Arora2018a}. Perhaps these factors suffice to explain the gap between human flexibility and that of deep models.\par 
In support of this hope, the results of \citet{Radford2019}, discussed above, offer a powerful example of the emergence of flexibility in machine learning. While natural language translation seems like a difficult machine-learning problem on its own \citep{Wu2016}, the results of \citet{Radford2019} show that a passable translation ability can emerge simply from training a large enough word-prediction model on a large enough corpus of webpages, and conditioning it on a few translation pairs. That is, the model is able to translate simply because translation is a systematic use of language, and so is implied by learning to predict language well enough. This result is surprising and promising. \par 
Relatedly, my collaborators and I showed that increasing various aspects of environmental richness and realism improved compositional language generalization in RL agents \citep{Hill2019a}. We found that making the agent more embodied (comparing an egocentric frame of reference to an allocentric frame in a 2D task) improved language generalization. We also found that switching from making a single-frame decision to behaving over time in an RL setting resulted in compositional generalization increasing from around 75\% to 100\% on the same abstract language generalization task. These results support the idea that more systematic generalization may emerge from more realistic and varied training regimes, unlike the extremely simplified settings that are sometimes used to critique deep learning models \citep[see e.g.][]{Lake2017}. Indeed, building language models with more embodiment, more realistic environments, and more integration and interaction between different systems may be key to more human-like language understanding \citep{McClelland2019}.\par
Furthermore, as I noted above, planning-like behavior can emerge in a model-free architecture, if it is is allowed to perform recurrent computations between actions \cite{Guez2019}. That is, abstract behaviors like planning do not necessarily need to be built into the system --- they can emerge from appropriate experience. This possibility is especially important when considering the rich cultural tools humans have built for transmitting knowledge and structuring the thinking of future generations (see above). How flexible and systematic could an unstructured deep neural network be after years of structured schooling? \par 
Unfortunately, the answer is still unclear --- these demonstrations of emergent flexibility have not yet reached human-level generality. While the model of Radford et al. exceeds non-trivial translation baselines, it is far from reaching the performance of a state-of-the-art model trained specifically for translation. These models in turn are not yet as sophisticated as humans at translation, perhaps in part for reasons outlined by \citet{McClelland2019}. \par
Thus, we simply do not know enough yet to determine whether the emergent effects of scale, curricula, and more realistic tasks are the only difference between the cognitive flexibility of humans and the comparatively inflexible intelligence displayed by neural networks, or whether we will need to incorporate symbol manipulation or another paradigm to capture the flexibility and generality of human intelligence. While the dominance of symbol-free deep networks in domains like natural language translation is promising, the ultimate answers will only come with further research and scale. In the next section, I discuss some abilities that are missing, at least from the scale of models that we currently have. In the remainder of the dissertation I will propose mechanisms that solve these problems at a feasible scale, and without requiring compositional symbol manipulation as an architectural axiom. \par 

\subsection{What's still missing from current models?}

Deep learning systems still lack some of the flexibility that humans have. Although deep learning systems can often learn a new task from few examples, most demonstrations of this rapid learning have involved sampling tasks that are within a dense region of the training task distribution.\footnote{See \citet{Chollet2019} for a detailed discussion of this limitation, and other related points.} Furthermore, humans have a great deal more flexibility than simply learning rapidly. For example, we can follow instructions to accomplish a novel task. \par
We can also adapt to a novel task without any data at all, if we know how it relates to prior tasks. For example, if we are told to try to lose at poker when we have previously been trying to win, it is easy for us to adapt, despite the fact that the new goal contradicts all our prior experience. with poker By contrast, it would be quite difficult for any contemporary reinforcement learning model to invert its value function. We can also achieve goals that are orthogonal to the original value function, for example trying to follow the motions of a meaningless background sprite. We can do these tasks reasonably well on our first try. That is, we can flexibly adapt our task representations in order to perform a new task zero-shot, based on its relationship to prior tasks. The meta-learning systems I have reviewed do not yet possess this flexibility. Even the emergent flexibility demonstrated by \citet{Radford2019} required them to condition the system on examples of translation --- it was not zero-shot. \par
Some of the zero-shot work I reviewed above does show the ability to perform new tasks without data, but if those models use the systematic relationships between the new tasks and the old, they do so only implicitly. That is, a system like that used in \citet{Hill2019a} may generalize to putting a red vase on the bed because the representations that emerge in its language system capture something about the color red and vases, and that allows it to generalize. However, it does not actively use its representations of prior tasks involving vases or red objects. It instead generates a representation for the new task completely from scratch. This approach seems fundamentally different from the approach humans would take when trying to switch to losing at poker -- rather than building our understanding of ``losing poker'' from scratch, using only implicit knowledge of poker, humans actively use our knowledge of poker, but use it in a systematically transformed way. I think this ability will be important for building more flexible deep-learning models.  \par
Thus, my main goal in this dissertation will be to propose a computational model that adapts to a new task by transforming a prior task representation, and to explore whether such a model provides a better model for adaptation than constructing a task representation from language alone. The model involves a system which learns to represent both data and tasks themselves in a shared latent space. It then learns to infer transformations of this space, which can be used both for performing basic tasks, but more critically for transforming task representations themselves. Transforming a prior task representation can allow the model to adapt to novel tasks zero-shot. By learning basic and more abstract transformations in a shared space, the model parsimoniously explains the human ability to adapt task representations, without needing to posit new systems for each new type of transformation.\footnote{This observation is related to the general point that humans have the ability to flexibly reason across levels of abstraction. We can relate between examples of a concept and what those examples imply about the overall concept, e.g. as counter-examples to universal properties. With training, we can even reason flexibly across complex hierarchies of abstraction, as when thinking about the mathematical concepts of numbers, sets, functions, and categories. We are able to recursively build abstractions on top of abstractions (although this is often a slow process). By contrast, deep learning models typically represent different levels of abstraction separately, e.g. at separate layers of a feed-forward architecture. This separation limits the flexibility of reasoning between levels of abstraction (the only available mapping is the canonical transformation given by the weights), and because the abstraction is built in to the architecture, it cannot be applied recursively (its depth is fixed). Furthermore, humans can reason both about data and the computations we perform over data, whereas most deep learning architectures restrict their knowledge of computations to weights to which they have no explicit access. This limits the flexibility and representational capacity of these networks. Addressing some of these limitations may be important for achieving more human-like intelligence from deep learning \citep[c.f.][]{Chollet2019}.} \par 
I will describe this model in detail in Chapter \ref{chapter:zero_shot_via_homm}. The model does not require building in task-specific knowledge or symbols, and is therefore extremely general, so I will demonstrate its effectiveness across a broad range of domains in the subsequent chapters. I will compare it to human zero-shot adaptation in simple card games in Chapter \ref{chapter:human}. I will also compare to the alternative approach of constructing a representation for the new task from language alone, without explicitly transforming prior tasks representations. I will then extend my approach to more complex tasks in Chapter \ref{chapter:extending}, including recognition of visual concepts and reinforcement learning. In Chapter \ref{chapter:timescales}, I will illustrate one reason why zero-shot adaptation is important -- it allows us to learn better once we begin the task, and make many fewer errors along the way to mastery. Finally, in Chapter \ref{chapter:conclusions}, I will return to the cognitive issues raised in this chapter, and discuss the broader implications of this dissertation.\par 


